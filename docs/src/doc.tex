\documentclass[12pt, a4paper]{article}

\usepackage[utf8]{inputenc}
\usepackage{cite}
\usepackage{amsmath}
\usepackage{amssymb}
\usepackage{url}
\usepackage[pdftex]{graphicx}
\usepackage{xifthen}
\usepackage{wrapfig}
\usepackage{abstract}
\usepackage[cm]{fullpage}
\usepackage{parskip}

\graphicspath{{../}}

%% Einbinden von Bildern
%%
%% Aufruf: \image{#1}{#2}{#3}{#4}
%% #1 ...: Bildbreite
%% #2 ...: Dateiname
%%         Der Dateiname mmit davor gesetztem „fig:“ ist gleichzeitig
%%         der Name ueber den das Bild referenziert werden kann.
%% #3 ...: Bildunterschrift
%% #4 ...: Position wo das Bild erscheinen soll
%% #5 ...: Label, Referenz
\newcommand{\image}[5]{
	\begin{figure}[#4]               %% Beginn der Gleitumgebung
		\centering                     %% Bild zentrieren
		\includegraphics[width=#1]{#2} %% Bild laden
		\caption{#3}                   %% Bildunterschrift
		\label{fig:#5}                 %% Label fuer Referenzierung
	\end{figure}                     %% Ende der Gleitumgebung
}

\title{Singing Very High Speed Integrated Circuit Hardware Description Language Board (S76D)}
\author{
        Kai Fabian \\
        Hasso-Plattner-Institut Potsdam\\
            \and
        Dominik Moritz \\
        Hasso-Plattner-Institut Potsdam\\
}
\date{\today}

\begin{document}
\maketitle

\image{0.7\textwidth}{render.png}{3D rendered image of the custom extension board used for this music player}{ht}{render}

\begin{abstract}
A VHDL implementation of the MMC SPI interface \ldots
\end{abstract}

\tableofcontents

\section{Introduction}

\section{Multimedia Card}

\subsection{Interface description}

Figure~\ref{fig:interface} shows an MMC card and Table~\ref{tab:interface} describes the pins of the MMC card.

\image{0.3\textwidth}{../mmc_pins.png}{MMC Interface}{h}{interface}

\begin{table}[hH]
    \begin{tabular}{|l|l|l|p{8cm}|}
    \hline
Pin. No.   & Name    & Description	    & Note for our implementation \\ \hline
1	       & DAT3	 & MMC: Chip select & Chip IO Pin, mmc\_cs, D7 \\
2	       & CMD	 & Command 	        & Chip IO Pin,  mmc\_mosi, D8, (MOSI – master out, slave in) \\
3	       & GND	 & Ground	        & Board ground \\
4	       & VDD	 & Voltage	        & Supply Voltage (+3.3 V) \\
5	       & CLK	 & Clock	        & Chip IO Pin,  mmc\_clock, D10 \\
6	       & GND	 & Ground	        & Board ground \\
7	       & DAT0	 & Data	            & Chip IO,  mmc\_miso, B4, (MISO - master in, slave out) \\
8	       & DAT1	 & Data	            & Pull up (Voltage via resistor) \\
9	       & DAT2	 & Data	            & Pull up (Voltage via resistor) \\
	\hline
    \end{tabular}
    \caption{MMC card pin description}
    \label{tab:interface}
\end{table}

\subsection{SPI Interface}

The SPI Interface uses two connections to the MMC. First the \emph{master out, slave in} to send commands to the MMC and the \emph{master in, slave out} to receive the results from the MMC. You can see the pins on the MMC in figure \ref{fig:interface}.

\image{0.8\textwidth}{../spi_mmc_command.pdf}{Sending a command to the MMC}{h}{command}

Sending commands requires you to stick to a predefined protocol that we will describe. A basic sequence for one command can be found in Figure~\ref{fig:command}. In Figure~\ref{fig:init} and~\ref{fig:read} you can find the sequence of commands we have to send to the card in order to initialize it or read a block.

Each command is a binary coded sequence in a format that is defined in the official reference manual. Figure~\ref{fig:command} should give you a basic idea of how the communication works. The start byte (SB) consists of 8 times HI (1). Before that, the card only sends 0s. The response token (RT) consists of 8 bits. Each bit indicates a certain error or a successful command. In case the command finished successfully, the card returns \texttt{0000 0001}.

\subsubsection{Initializing the card}

The MMC cards needs to be initialized before it can be used to read data. According to the specs, one need to send a \texttt{CMD0} and \texttt{CMD1} after at least 74 clock cycles to synchronize the card clock. The initialization process is complete when the card responds with \texttt{0000 0000} (busy) instead of \texttt{0000 0001} (ready).

The whole initialization procedure is shown in Figure~\ref{fig:init}.

\image{0.5\textwidth}{../flow_init_mmc.pdf}{Flow for initializing}{h}{init}

\subsubsection{Reading data}

Reading from the MMC card is shown in Figure~\ref{fig:read}. The command \texttt{CMD17} to reads 1 block of data from the MMC card starting at the address specified in the argument field. The size of the block that is read has to be defined with a command \texttt{CMD16} (which is not shown here).

\image{0.5\textwidth}{../flow_read_mmc.pdf}{Flow for reading}{h}{read}

\subsection{Implementation of Communication}

Figure~\ref{fig:implementation} shows the finite state machine of our implementation in VHDL. \texttt{\{n\}} means that the number preceding the brackets is repeated $n$ times.

\image{1.0\textwidth}{../mmc_controller.pdf}{Controller VHDL FSM}{h}{implementation}

\subsection{Most important MMC SPI commands}

Table~\ref{tab:commands} shows the most important command that we used in our implementation.

\begin{table}[hH]
    \begin{tabular}{|l|l|p{8cm}|}
    \hline
Command    & Name                & Description            \\ \hline
CMD0       & GO\_IDLE\_STATE     & Reset the card. If CS == 0, the card goes into SPI mode. \\
CMD1       & SEND\_OP\_COND      & initialize the card. The command has to be sent repeatedly until the response changes from \texttt{0000 0000} (busy) to \texttt{0000 0001} (ready). \\
CMD16      & SET\_BLOCKLEN       & Set the block size. Send the block size as the argument. \\
CMD17      & READ\_SINGLE\_BLOCK & Read one block as specified in \texttt{CMD16}. Send the start address as the argument. \\
    \hline
    \end{tabular}
    \caption{MMC SPI commands used in this implementation.}
    \label{tab:commands}
\end{table}

More details about the SPI commands, responses and timings can be found in the MMC specification.

\listoffigures

\end{document}
This is never printed